%%%%%%%%%%%%%%%%%%%%%%%%%%%%%%%%%%%%%%%%%
% Short Sectioned Assignment
% LaTeX Template
% Version 1.0 (5/5/12)
%
% This template has been downloaded from:
% http://www.LaTeXTemplates.com
%
% Original author:
% Frits Wenneker (http://www.howtotex.com)
%
% License:
% CC BY-NC-SA 3.0 (http://creativecommons.org/licenses/by-nc-sa/3.0/)
%
%%%%%%%%%%%%%%%%%%%%%%%%%%%%%%%%%%%%%%%%%

%----------------------------------------------------------------------------------------
%	PACKAGES AND OTHER DOCUMENT CONFIGURATIONS
%----------------------------------------------------------------------------------------

\documentclass[paper=a4, fontsize=11pt]{scrartcl} % A4 paper and 11pt font size

\usepackage[T1]{fontenc} % Use 8-bit encoding that has 256 glyphs
\usepackage{fourier} % Use the Adobe Utopia font for the document - comment this line to return to the LaTeX default
\usepackage[english]{babel} % English language/hyphenation

\usepackage{lipsum} % Used for inserting dummy 'Lorem ipsum' text into the template

\usepackage{sectsty} % Allows customizing section commands
\allsectionsfont{ \normalfont\scshape} % Make all sections centered, the default font and small caps
\usepackage{cite}
\usepackage{listings}

%%%%%---------------------------%%%%%%%%%%%
\usepackage{fancybox}
\usepackage{graphicx}
\usepackage{pdfpages}
\usepackage{color}
\usepackage{epstopdf}
\usepackage[margin=1in, vmargin=1in]{geometry}
\usepackage{mathtools}
\usepackage{float}
\usepackage{listings}
\usepackage{verbatim}
\usepackage{booktabs}
\usepackage{tabularx}
\usepackage{longtable, array}
\usepackage{movie15}
\usepackage{hyperref}
\usepackage{subcaption}
\usepackage{enumerate}
\usepackage{hyperref}
\usepackage{bashful}

\usepackage{multicol}
\usepackage{booktabs} % For \toprule, \midrule and \bottomrule
\usepackage{siunitx} % Formats the units and values
\usepackage{pgfplotstable} % Generates table from .csv

%%%%%%%%%%%%%%---------------%%%%%%%%


\usepackage{fancyhdr} % Custom headers and footers
\pagestyle{fancyplain} % Makes all pages in the document conform to the custom headers and footers
\fancyhead{} % No page header - if you want one, create it in the same way as the footers below
\fancyfoot[L]{} % Empty left footer
\fancyfoot[C]{} % Empty center footer
\fancyfoot[R]{\thepage} % Page numbering for right footer
\renewcommand{\headrulewidth}{0pt} % Remove header underlines
\renewcommand{\footrulewidth}{0pt} % Remove footer underlines
\setlength{\headheight}{13.6pt} % Customize the height of the header

\numberwithin{equation}{section} % Number equations within sections (i.e. 1.1, 1.2, 2.1, 2.2 instead of 1, 2, 3, 4)
\numberwithin{figure}{section} % Number figures within sections (i.e. 1.1, 1.2, 2.1, 2.2 instead of 1, 2, 3, 4)
\numberwithin{table}{section} % Number tables within sections (i.e. 1.1, 1.2, 2.1, 2.2 instead of 1, 2, 3, 4)

\setlength\parindent{0pt} % Removes all indentation from paragraphs - comment this line for an assignment with lots of text

%%%%%%%%CODE INPUT STYLES%%%%%%%%%%%%%
\lstdefinestyle{BashInputStyle}{
  language=bash,
  firstline=2,% Supress the first line that begins with `%`
  basicstyle=\small\sffamily,
  numbers=left,
  numberstyle=\tiny,
  numbersep=3pt,
  frame=tb,
  columns=fullflexible,
  backgroundcolor=\color{yellow!20},
  linewidth=0.9\linewidth,
  xleftmargin=0.1\linewidth
}

\lstdefinestyle{BashOutputStyle}{
  basicstyle=\small\ttfamily,
  numbers=none,
  frame=tblr,
  columns=fullflexible,
  backgroundcolor=\color{blue!10},
  linewidth=0.9\linewidth,
  xleftmargin=0.1\linewidth
}


% settings for listings
\lstset{
    basicstyle=\scriptsize,
    numbers=left,
    numberstyle=\scriptsize,
    stepnumber=1,
    numbersep=5pt,
    showspaces=false, % don't show spaces by adding underscores
    showstringspaces=false, % don't underline spaces in strings
    showtabs=false, % don't show tabs with underscores
    frame=shadowbox,
    tabsize=4,
    captionpos=b,
    breaklines=true,
    breakatwhitespace=false,
    keywordstyle=\color{blue!70},
    commentstyle=\color{red!50!green!50!blue!50},
    rulesepcolor=\color{red!20!green!20!blue!20},
    numberbychapter=false,
    stringstyle=\ttfamily %typewriter type for strings
}
\hypersetup{
colorlinks=true,
linkcolor=black,
citecolor=black,
urlcolor=black
}

%----------------------------------------------------------------------------------------
%	TITLE SECTION
%----------------------------------------------------------------------------------------

\newcommand{\horrule}[1]{\rule{\linewidth}{#1}} % Create horizontal rule command with 1 argument of height

\title{	
\normalfont \normalsize 
\textsc{Introduction to Web Science- Fall 2014} \\ [25pt] % Your university, school and/or department name(s)
\horrule{0.5pt} \\[0.4cm] % Thin top horizontal rule
\huge Assignment Ten \\ % The assignment title
\horrule{2pt} \\[0.5cm] % Thick bottom horizontal rule
}

\author{Sybil Melton} % Your name

\date{\normalsize\today} % Today's date or a custom date

\begin{document}

\maketitle % Print the title
\newpage
\tableofcontents
%\listoffigures
\listoftables
\lstlistoflistings
\newpage
%----------------------------------------------------------------------------------------
%	TASK 1
%----------------------------------------------------------------------------------------

\section{Blog Retrieval}
Choose a blog or a newsfeed (or something similar as long as it has
an Atom or RSS feed).  It should be on a topic or topics of which you
are qualified to provide classification training data.  In other words,
choose something that you enjoy and are knowledgable of.  Find a feed
with at least 100 entries.

Create between four and eight different categories for the entries
in the feed.

Download and process the pages of the feed as per the week 12 
class slides.

\subsection{Solution}
To find a blog, I used the ``next blog'' functionality to randomly choose one until I found one with a technology topic and more than 100 entries.
The particular blog I chose is titled ``The Invisible Things Lab's blog.''
The description is Kernel, Hypervisor, Virtualization, Trusted Computing and other system-level security stuff, which I have knowledge
of from work and my studies at ODU.
I used requests, feedparser, and BeautifulSoup to get the blog pages and parse the feed.
I retrieved the first four pages and saved the entries in a list, to be filtered later.
The content still had some markup language and illegal XML characters, so I used the same functions I created for assignment nine
to remove them.
Listing \ref{code:blog} is the code used to accomplish the first part of this task.\cite{bib:collective}

\lstset{
    caption={Retrieve Blog},
     label=code:blog
}
\lstinputlisting[language=Python, firstline=10, lastline=47]{classify.py}

As I read the entries, I chose announcements, security, hardware, os, and paper as the categories.  Announcements was for general announcements.  Security was for any entry about cybersecurity.  Hardware entries discussed chipset and DRAM.  Os handled any general articles about operating systems that did not fit into security, and paper handled any blogs that just posted white papers or slides from conferences.

%----------------------------------------------------------------------------------------
%	TASK 2
%----------------------------------------------------------------------------------------
\section{Classification}
Manually classify the first 50 entries, and then classify (using
the fisher classifier) the remaining 50 entries. Report the cprob()
values for the 50 titles as well.  From the title or entry itself,
specify the 1-, 2-, or 3-gram that you used for the string to
classify.  Do not repeat strings; you will have 50 unique strings.

Create a table with the title, the string used for classification,
cprob(), predicted category, and actual category.

\subsection{Solution}
My program set up the fisher classifier and database.
The database was ``feed.db'' and is included with this report.
Then feedfilter was used to manually classify the first 50 entries, train the classifier, and guess a classification for the remaining 50.
Listing \ref{code:setup} is the code used to accomplish the next task.\cite{bib:collective}

\lstset{
    caption={Set Up Classifier},
     label=code:setup
}
\lstinputlisting[language=Python, firstline=49, lastline=52]{classify.py}

I modified feedfilter.py to classify the entries.  
I changed the read function to accept my list of entries, so the list could be split between the first and second 50 entries.
The `summary' token was changed to the content value, because I found some entries were cut off because of the length.
The title and content were concatenated before sent to the classifier.
As the loop progressed, it added my classifier and manually assigned categories to the list for each entry.
Then the train function from docclass.py was called and was not modified.
Listing \ref{code:first} is the modified code in feedfilter.py to accomplish this.\cite{bib:collective}

\lstset{
    caption={First 50 Classification},
     label=code:first
}
\lstinputlisting[language=Python, firstline=7, lastline=23]{feedfilter.py}

Next the second 50 entries had to classified by me and the classifier.
Additionally, the cprob had to be saved, so I modified docclass.py classify function to return it when a guess was given.
This allowed me to save both the predicted category and cprob for each in the entries list.
Listing \ref{code:second} is the additional code in feedfilter.py to accomplish this.\cite{bib:collective}
\lstset{
    caption={Second 50 Classification},
     label=code:second
}
\lstinputlisting[language=Python, firstline=25, lastline=44]{feedfilter.py}

In order to create the table, the title, classifier strings, predicted category, actual category, and cprob were sent to ``table.txt.''
The output file was then converted to ``table.tex,'' in order to create table \ref{tab:data}.
Listing \ref{code:table} is the code to output the data.\cite{bib:collective}
\lstset{
    caption={Table Data},
     label=code:table
}
\lstinputlisting[language=Python, firstline=54, lastline=63]{classify.py}



 \begin{small}
 \begin{longtable}{|p{5cm}|l|l|l|r|}
 \caption{Classifier Data} \label{tab:data} \\ \hline
\textbf{\textsc{Title}}&\textbf{\textsc{Classifier}}&\textbf{\textsc{Predicted}}&\textbf{\textsc{Actual}}&\textbf{cprob()}\\ \hline\hline
\endhead
Qubes R3/Odyssey initial source code release&Join us&  &announcements&  \\ \hline 
Announcing Qubes OS Release 2!&bug fixes&  &announcements&  \\ \hline 
Physical separation vs. Software compartmentalization&topics discussed&  &paper&  \\ \hline 
Qubes OS R2 rc2, Debian template, SSLed Wiki, BadUSB, and more...&updates&  &announcements&  \\ \hline 
Qubes OS R2 rc1 has been released!&improvements&  &announcements&  \\ \hline 
Shattering the myths of Windows security&Windows security&  &security&  \\ \hline 
Qubes R2 Beta 3 has been released!&beta&  &announcements&  \\ \hline 
Windows 7 seamless GUI integration coming to Qubes OS!&Support Tools&  &os&  \\ \hline 
Thoughts on Intel's upcoming Software Guard Extensions (Part 2)&memory bus&  &hardware&  \\ \hline 
Thoughts on Intel's upcoming Software Guard Extensions (Part 1)&DRAM&  &hardware&  \\ \hline 
Qubes OS R3 Alpha preview: Odyssey HAL in action!&configuration template&  &announcements&  \\ \hline 
Introducing Qubes Odyssey Framework&hardcoded&  &announcements&  \\ \hline 
Qubes 2 Beta 2 has been released!&drivers&  &announcements&  \\ \hline 
Converting untrusted PDFs into trusted ones: The Qubes Way&PDFs&  &paper&  \\ \hline 
Qubes 2 Beta 1 with initial Windows support has been released!&windows support&  &os&  \\ \hline 
How is Qubes OS different from...&linux BSD&  &os&  \\ \hline 
Introducing Qubes 1.0!&introducing Qubes&  &announcements&  \\ \hline 
Qubes 1.0 Release Candidate 1!&release candidate&  &announcements&  \\ \hline 
Some comments on "Operation High Roller"&two-factor authentication&  &security&  \\ \hline 
Windows support coming to Qubes!&coming to&  &announcements&  \\ \hline 
Qubes Beta 3!&installation guide&  &announcements&  \\ \hline 
Thoughts on DeepSafe&firewall&  &security&  \\ \hline 
Trusted Execution In Untrusted Cloud&confidentiality&  &security&  \\ \hline 
Exploring new lands on Intel CPUs (SINIT code execution hijacking)&processors&  &hardware&  \\ \hline 
Playing with Qubes Networking for Fun and Profit&infrastructure&  &os&  \\ \hline 
Qubes Beta 2 Released!&proud to announce&  &announcements&  \\ \hline 
Anti Evil Maid&disk encryption&  &security&  \\ \hline 
Interview about Qubes OS&interview about&  &announcements&  \\ \hline 
My SSTIC 2011 slides&security&  &security&  \\ \hline 
From Slides to Silicon in 3 years!&presentation&  &announcements&  \\ \hline 
USB Security Challenges&software attacks&  &security&  \\ \hline 
(Un)Trusting the Cloud&encrypted&  &security&  \\ \hline 
The App-oriented UI Model and its Security Implications&security by isolation&  &security&  \\ \hline 
Following the White Rabbit: Software Attacks Against Intel VT-d&publish&  &paper&  \\ \hline 
The Linux Security Circus: On GUI isolation&sniff all&  &security&  \\ \hline 
Why the US "password revolution" won't work&multi-factor authentication&  &security&  \\ \hline 
Qubes Beta 1 has been released!&new features&  &announcements&  \\ \hline 
Partitioning my digital life into security domains&compromise&  &security&  \\ \hline 
My documents got lost/stolen [offtopic]&SSL cert&  &security&  \\ \hline 
Update on Qubes&Stay tuned&  &announcements&  \\ \hline 
Qubes Alpha 3!&milestone&  &announcements&  \\ \hline 
ITL is hiring!&looking to hire&  &announcements&  \\ \hline 
On Thin Clients Security&desktop security&  &security&  \\ \hline 
(Un)Trusting your GUI Subsystem&depriviliged&  &security&  \\ \hline 
Qubes, Qubes Pro, and the Future...&the future&  &announcements&  \\ \hline 
The MS-DOS Security Model&attacker exploiting&  &security&  \\ \hline 
Skeletons Hidden in the Linux Closet: r00ting your Linux Desktop for Fun and Profit&malicious&  &security&  \\ \hline 
Qubes Alpha 2 released!&are here&  &announcements&  \\ \hline 
Disposable VMs&non-sensitive&  &security&  \\ \hline 
On Formally Verified Microkernels (and on attacking them)&verification attempts&  &security&  \\ \hline 
Evolution&  &announcements&announcements&1.0\\ \hline 
Remotely Attacking Network Cards (or why we do need VT-d and TXT)&  &announcements&security&1.0\\ \hline 
Introducing Qubes OS&  &announcements&announcements&1.0\\ \hline 
Priorities&  &security&security&1.0\\ \hline 
Another TXT Attack&  &announcements&security&1.0\\ \hline 
Evil Maid goes after TrueCrypt!&  &announcements&security&1.0\\ \hline 
Intel Security Summit: the slides&  &announcements&paper&1.0\\ \hline 
About Apple's Security Foundations Or Lack Of Thereof...&  &security&security&1.0\\ \hline 
PDF signing and beyond&  &security&security&1.0\\ \hline 
Vegas Toys (Part I): The Ring -3 Tools&  &announcements&security&1.0\\ \hline 
Black Hat 2009 Slides&  &announcements&paper&1.0\\ \hline 
Interview&  &announcements&announcements&0.983\\ \hline 
Virtualization (In)Security Training in Vegas&  &announcements&announcements&1.0\\ \hline 
Quest to The Core&  &announcements&announcements&1.0\\ \hline 
More Thoughts on CPU backdoors&  &security&security&1.0\\ \hline 
Thoughts About Trusted Computing&  &announcements&security&1.0\\ \hline 
Trusting Hardware&  &security&security&1.0\\ \hline 
The Sky Is Falling?&  &security&security&1.0\\ \hline 
Attacking SMM Memory via Intel  CPU Cache Poisoning&  &announcements&security&1.0\\ \hline 
Independent Attack Discoveries&  &announcements&security&1.0\\ \hline 
Attacking Intel TXT: paper and slides&  &announcements&paper&0.997\\ \hline 
Nesting VMMs, Reloaded.&  &announcements&announcements&0.997\\ \hline 
Closed Source Conspiracy&  &security&security&1.0\\ \hline 
Why do I miss Microsoft BitLocker?&  &security&security&1.0\\ \hline 
Attacking Intel Trusted Execution Technology&  &security&security&1.0\\ \hline 
Microsoft executive "rebuts" our research!&  &announcements&security&1.0\\ \hline 
Xen 0wning Trilogy: code, demos and q35 attack details posted&  &announcements&security&1.0\\ \hline 
The three approaches to computer security&  &security&security&1.0\\ \hline 
Teamwork Crediting&  &security&security&1.0\\ \hline 
Intel patches the Q35 bug&  &announcements&announcements&1.0\\ \hline 
Attacking Xen: DomU vs. Dom0 consideration&  &announcements&security&1.0\\ \hline 
Our Xen 0wning Trilogy Highlights&  &security&announcements&1.0\\ \hline 
0wning Xen in Vegas!&  &announcements&announcements&1.0\\ \hline 
Rafal Wojtczuk joins Invisible Things Lab&  &announcements&announcements&1.0\\ \hline 
1984?&  &announcements&announcements&0.999\\ \hline 
Vegas Training 2008&  &announcements&announcements&1.0\\ \hline 
Research Obfuscated&  &security&hardware&1.0\\ \hline 
The Most Stupid Security News Ever&  &security&security&1.0\\ \hline 
The RSA Absurd&  &security&security&1.0\\ \hline 
Kick Ass Hypervisor Nesting!&  &security&os&1.0\\ \hline 
Razor-Thin Hypervisors&  &announcements&announcements&1.0\\ \hline 
Thoughts On Browser Rootkits&  &announcements&security&1.0\\ \hline 
Tricky Tricks&  &announcements&security&1.0\\ \hline 
Virtualization Detection vs. Blue Pill Detection&  &announcements&announcements&1.0\\ \hline 
We're ready for the Ptacek's challenge!&  &announcements&announcements&1.0\\ \hline 
Invisible Things Lab, Bitlocker/TPM bypassing and some conference thoughts&  &announcements&security&1.0\\ \hline 
Understanding Stealth Malware&  &announcements&security&1.0\\ \hline 
The Human Factor&  &security&security&1.0\\ \hline 
The Game Is Over!&  &security&security&1.0\\ \hline 
Handy Tool To Play with Windows Integrity Levels&  &announcements&os&1.0\\ \hline 

\end{longtable}
\end{small}


%----------------------------------------------------------------------------------------
%	TASK 3
%----------------------------------------------------------------------------------------
\section{Performance}
Assess the performance of your classifier in each of your categories
by computing precision, recall, and F1.
Note that the definitions
of precisions and recall are slightly different in the context of
classification; see:

http://en.wikipedia.org/wiki/Precision\_and\_recall\#Definition\_.28classification\_context.29

and

http://en.wikipedia.org/wiki/F1\_score

\subsection{Solution}
For a classifier, the expected prediction will be positive if labelled correctly and negative if not.
The observation will be true or false, as determined by external judgement.
A true positive (TP) is an entry labelled correctly.
A false negative (FN) would be an entry that was not labelled, but it should have been.  
This classifier did not have any FN.
A false positive (FP) was incorrectly labelled.\cite{bib:class}
Precision, or positive prediction value, was calculated as:
\[
\textrm{Precision}  = \frac{tp}{ tp + fp}
\]
The recall, or sensitivity, was calculated as:
\[
\textrm{Recall}  = \frac{tp}{tp + fn}
\]
The F1 score, or accuracy, was calculated as:
\[
F_1  = 2 * \frac{\textrm{precision}*\textrm{recall}}{\textrm{precision} + \textrm{recall}}
\]

Listing \ref{code:calc} is the code used to perform the calculations. \cite{bib:class},\cite{bib:f1}

\lstset{
    caption={Calculations},
     label=code:calc
}
\lstinputlisting[language=Python, firstline=67, lastline=81]{classify.py}

The classifier precision was only .58 and because there were no FN, the recall score was 1.
This provided an F1 score of 0.734.
I believe this was due to the lack of training for the paper and os categories.  
There were far fewer entries that belonged here, so they were not classified correctly by the classifier.
Perhaps if there had been more trained entries, it would have been higher.


Finally, all entries and data were output to ``entries.txt'' to be included in this report.\cite{bib:lfile}
Listing \ref{code:classify} in section \ref{sec:code} is the entire classify.py program and listing \ref{code:filter}, inthe same section
is feedfilter.py.
The file docclass.py, as used,  is included with this report.
\newpage
\section{Python Code}
\label{sec:code}
%%% Code Listing%%%%%

\lstset{
    language=python,
    caption={Complete Classifier},
     label=code:classify
}

\lstinputlisting{classify.py}

\lstset{
    language=python,
    caption={Modified Feed Filter},
     label=code:filter
}

\lstinputlisting{feedfilter.py}
\newpage 


\bibliography{references}{}
\bibliographystyle{plain}
\end{document}